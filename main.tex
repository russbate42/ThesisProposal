\documentclass[12pt]{article}

\usepackage{import} % for importing alternative to include\input
%\usepackage{geometry, fancyhdr, graphicx, subfigure}
\usepackage{geometry, fancyhdr, graphicx, subcaption}
\usepackage{physics}
\usepackage{amsmath}
\usepackage{tikz}


% This file contains specific commands for the geometry
% package and customizations


\graphicspath{{figures/}}
\usepackage{xparse}
\usepackage{wrapfig}
\usepackage{multirow}
\usepackage{multicol}
\usepackage{mathrsfs}
% \usepackage{glossaries}
% % This file contains specific customizations for the fancyhdr
% package




\usepackage{silence}
\WarningsOff[lipsum]
\WarningFilter[lipsum]{lipsum}{Package lipsum Warning}
\usepackage{lipsum}
\import{./}{dummy-text-generator.tex}

% Math command
\newcommand{\pd}{\partial}

% Only with LaTeX-Physics
\renewcommand{\vb}{\vectorbold}

% For glossary without glossaries package
\newcommand{\term}[2]{\textbf{#1}: #2}
\newcommand{\acronym}[2]{\textbf{#1} -- #2}

\begin{document}
\ActivateWarningFilters[lipsum]
\import{./}{title-page.tex}

\newpage

\newpage
\section{Introduction}
This article is a thesis proposal for a Doctorate of Philosophy from the
Physics and Astronomy Department at the University of British Columbia. My
thesis uses data from the ATLAS experiment, one of the four main experiments at
the Large
Hadron Collider at CERN.

For the past four years I have been working with the ATLAS experiment.

%% Maybe get rid of this altogether?

\section{The ATLAS Experiment}
The ATLAS experiment is a general purpose particle detector at the Large Hadron
Collider (LHC) at the European Center for Nuclear Physics Research (CERN). It
stands for \textbf{A} \textbf{T}oroidal \textbf{L}HC \textbf{A}pparatu\textbf{S}.
The detector is named after the toroidal magnet system which is integral to the
design and plays an important role in physical observations. ATLAS is designed
to be a general purpose particle detector to observe collisions from the full
LHC beam energy and intensity. It's dimensions are 44 meters across, has a
diameter of 25 meters, and is over 7000 tonnes \cite{Aad_2024}. 
\begin{wrapfigure}{l}{0.42\textwidth}
    \centering
    \includegraphics[width=0.4\textwidth]{Aerial_view_of_the_area_where_the_LHC_ring_is_located}
    \caption{Aerial view of the LHC}
    \label{fig:aerial_view}
\end{wrapfigure}
As a general purpose particle detector it is a composite of many different
subdetector systems, each with their own physics objectives in mind. A brief
overview of important subsystems are provided in the following subsections. The
detector covers nearly the entire solid angle around the collision point
\cite{The_ATLAS_Collaboration_2008}. 

The LHC beam can supply protons and heavy ion nuclei at energies up to $13.6$
TeV \ref{tab:data_taking} \cite{Aad_2024}. Particles are injected into the LHC
ring in bunches from the SPS beam. Bunches are held in place with Radio
Frequency (RF) electric fields, with a given spacing. During comissioning and
testing, bunches can be partially filled, and the LHC has a specified
\textit{bunch fill pattern} \cite{}. During data taking however, the LHC is nominally
operating with a full bunch fill. During beam preparations, when the fill is
complete, the beam energy is slowly ramped up to target operational energies.
Bunches are guided around the LHC ring with dipole magnets \cite{}. At
operational energies, particles are highly lorentz boosted, and have a bunch
crossing spacing of $25$ ns. Collisions occur at the crossing point located at
the center of the detector, and are subject to \textit{beam optics}. Beam optics
govern the number of collisions per bunch crossing through the density of
bunches and accuracy of the collision point. Higher beam collimation leads to
higher instantaneous luminosity. This is parameterized at the LHC by $\beta*$
\cite{Aad_2024}. The many collisions that occur during a single bunch crossing
is known as \textit{in time pile-up} and is paramaterized by $<\mu>$. This
depends on beam conditions during each data taking year. Different data taking
years is shown in \ref{fig:pileup}. 

\subsection{Detector Subsystems}
\subsubsection{Tracking and Inner Detector}
Tracking of charged particle tracks is done as a whole with the Inner Detector
(ID). The inner detector is a composite of subdetector systems. The inner
detector sits inside of a superconducting solenoid, and holds a magnetic field
of 2 Tesla, where the field lines are parallel to the beam axis. It has
psuedorapidity coverage of $|\eta| < 2.5$. The subsystems involved are the
Silicon Pixel detector, the Insertable B Layer (IBL) housed inside the Pixel,
the Semiconductor Tracker (SCT), and the Transition Radiation Tracker (TRT).

The Silicon Pixel detector along with the IBL has the highest granularity of
tracking and can provide up to four measurements per track. There are three
barrel layers and three endcap layers, and the IBL provides the fourth inner
layer. It has $r, \phi, z$ spatial measurement capabilities with a granularity
of $50 \mu m \times 400 \mu m$. The Pixel has approximately 80 million readout
channels. Along with the IBLs 12 million, there are 92 million readout channels
combined \cite{Aad_2024}.

The SCT is a silicon microstrip tracker which provides up to eight hits per
track, which results in four spatial measurements. In the barrel, one layer of
strips is parallel to the beam axis, and provides $r,\phi$ measurements. In
conjunction, there are small angle (40 mrad) stereo strips which provide
$r,\phi,z$ measurements. In the endcap, it is constructed with radially oriented
strips with again 40 mrad stereo strips, with a mean pitch of $80 \mu m$. In
total, the SCT has approximately 6.3 million readout channels.

Finally we have the TRT. It is constructed with gas filled straw tubes,
interleaved with transition radiation material. There are (look up)
approximately 300k proportional drift tubes. The TRT provides additional track
measurements at $|\eta| \leq 2.0$ and $p_T<0.5$ GeV. Due to the radial distance,
it aids in extended track fitting. A track can average 30 $r,\phi$ points with
130 $\mu m$ resolution. In addition, it aids in electron identification (eID).
This is achieved by measuring the fraction of hits above a higher energy deposit
threshhold corresponding to the transition radiation \cite{Aad_2024}.

\subsubsection{Calorimetry}

The ATLAS Calorimetry covers the entire psuedorapidity range of $|\eta| < 4.9$.
It is a sampling calorimeter, which is intended to entirely stop electromagnetic
showers and particles that are strongly interacting, as well as measure the
incident energy as accurately as possible. In addition, the calorimetry is
segmented in $r, \phi, z$ to provide position measurements. There are two
variants of sampling calorimetry technology used - Liquid Argon (LAr) and Tiled
Steel and Scintillator material. The LAr Calorimetry is of two varieties - Lead
Liquid Argon, and Copper Liquid Argon. A full schematic is shown in figure
\ref{fig:calo_schematic}

The Copper Liquid Argon is in the outer ends of the calorimeter - the Hadronic
End Cap (HEC), and is composed of four layers. These layers are segmented
further into radial petals, segmented by $r,\phi$. The inside end caps are three
layers of Lead LAr, the ElectroMagnetic End Caps (EMEC). Both end caps are
housed inside the same cryostat unit. Psuedorapidity ranges from $|\eta|>3.2$
are covered by both end caps.

The inner barrel is again Lead LAr, and is an electromagnetic calorimeter, thus
the ElectroMagnetic Barrel (EMB). It is constructed from three concentric
cylinders, and segmented in $\phi, z$. EMB1 has fine granularity in $z$,
providing precise measurements of psuedorapidity. Imporantly, this is used for
electron identification. The EMB is housed in it's own separate cryostat unit.

The outer barrel is the Tile Scintillation calo. Steel was chosen as the
absorber as it is cheap and has relatively high density in order to increase the
strong force interaction length. Charged particles emitted from primarily
hadronic showers, and electromagnetic showers as a secondary mechanism interact
with a scintillating plastic tiled between the steel plates. The short
wavelength photons emitted from the scintillators are coupled to wavelength
shifting optical fibers, which are ganged together into optical read out
modules. The current produced from the modules is a highly non-linear function
of the input light in a highly non-linear way. This is in part what makes
hadronic calorimetry within ATLAS challenging
\cite{The_ATLAS_Collaboration_2008}\cite{ml4p}. The Tile calo is covered by
$|\eta|<1.7$

There is some compromise for the calorimetry - primarily hadronic - in the
transition region between the barrel and the end caps due to geometry and design
limitations. A full break down of the rapidity coverage by the calorimeter is
shown in \ref{fig:calo_rapidity}.

\subsubsection{Muon System}

The muon system in ATLAS provides tracking and triggering on muon events. As
muons are heavy, the electromagnetic energy losses are small. Thus muons of
sufficient energy pass through the detector without fully showering. Therefore
the muon system is the outermost component of ATLAS and form the large outside
structure. The muon system is composed of two main systems. The Muon Spectometer
(MS) and the end caps. 

The muon spectrometer is the outer ``barrel''. The structure is formed with
superconducting air core magnets, which produce field lines that are
approximately concentric around the barrel. This was a design choice to produce
fields which are approximately orthogonal to the direction of travel. Thus, this
enables the measurement of track $p_T$ via bending as a result of the magnetic
field. The range of $|\eta|<1.4$ is fully covered throught the barrel toroid.
measures $p_T$ through the magnetic deflection of tracks. The outer toroid forms
the large outside component of ATLAS and provides psuedorapidity coverage of
$|\eta|<2.7$.

Paragraph about wheels

\begin{itemize}
    \item Thin Gap Chambers (TGCs)
    \item MDT (?)
    \item Resistive Plate Chambers (RPCs)
    \item 
\end{itemize}

During Long Shutdown 2, the inner wheels were replaced with the ``New Small
Wheels'' (NSW). This upgrade provides tracking measurements $1.3<|\eta|<2.7$. It
is constructed with two small tric TGCs (sTGCs), and Micromegas micromesh
technology. Both are capable fo tracking measurements, and lowered the $p_T$
resolution. In addition, this upgrade significantly reduced the read
out, which was necessary for higher luminosity data taking. Namely, the L1 rate
was reduced - which will be covered in Section \ref{sec:tdaq}.

\subsubsection{ATLAS TDAQ}\label{sec:tdaq}
Recording is handled by the Trigger and Data AQuisition system (TDAQ). The
amount of data ATLAS can record exceeds the ability process, stream, and store
it. Furthermore, not all bunch collisions produce physics of interest to
analysis groups. This is why we require a trigger system. This system is
designed to efficienctly make decisions of what events to record. This is done
in two stages, the level one trigger (L1), and the High Level Trigger (HLT). L1
is implemented with a combination of hardware and firmware, carried out in
successive stages of read out boards. It has a maximum allowable latency for the
electronics of $2.5 \mu\text{s}$ per event. L1 often uses simplifications to
produce low resolution physics objects. For example, the L1 Calorimetry groups
together electrodes to produce very low resolution jets. Given a bunch group
spacing of 25 ns, this yields a crossing of 40 MHz. The Read Out System (ROS) is
designed to handle stream rates of 100 kHz, with an additional 20\% contingency.
Thus the overall requirement of the L1 trigger system is to take events at 40
MHz, and selectively trigger on interesting physics events at a rate of 100 kHz.

The HLT decisions are made with a CPU farm of (fix me) thousands of CPUs. The
goal is to reconstruct physics observables as closely as possible to offline
selection and reconstruction. The HLT event processing time for an instantaneous
luminosity of $\mathscr{L} = 2\times 10^{34}\;cm^{-2}s^{-1}$ is approximately
400 ms.  Average event sizes at a pileup of $<\mu> \approx 60$ are 2.1 MB. The
HLT has a maximum allowable rate of 5 kHz. The limitation on recording is the
data throughput, at 8 Gb/s. Thus the goal is to have an average HLT output rate
of 3 kHz, which produces 6.3 GB/s of output, with a maximum allowable output of
5 kHz.

A schematic of the TDAQ system is given in \ref{fig:tdaq_schematic}.


\subsection{Data Taking and Conditions}
The ATLAS detector has been operational since 2008, undergoing annual data
taking runs and shutdowns, as well as long shut-down periods (LS1 and LS2).
Currently, the physics program with ATLAS is finalizing it's third data taking
run - Run 3. This is summarized in the following table.
\begin{table}[h]
    \centering
    \begin{tabular}{ l c c c }
        % \noalign{\vskip0pt}\hline\noalign{\vskip3pt}
        & Years & Integrated Luminosity [$\text{fb}^{-1}$] & $\sqrt{s}$ Energy [TeV] \\ 
        \noalign{\vskip0pt}\hline\noalign{\vskip3pt}
        \multirow{2}{*}{Run 1} & \multirow{2}{*}{$2009-2013$} & 5 & 7 \\  
         &  & 21 & 8 \\
        LS1 & $2013-2015$ & - & - \\
        Run 2 & $2015-2018$ & 147 & 13 \\
        LS2 & $2019-2022$ & - & - \\
        Run 3 & $2022-2025$ & 250 & 13.6 \\
        \hline
    \end{tabular} 
    \caption{Summary of data taking with the ATLAS detector \cite{Aad_2024}.}
    \label{tab:data_taking}
\end{table}
During Run 2, ATLAS reached a peak \textit{instantaneous} luminosity of
$\mathscr{L} = 2.1\times 10^{34} \;\text{cm}^{-2}\text{s}^{-1}$. Average
interactions per bunch crossing are 33.7 although this changed year over year
\ref{fig:bunch_crossings}. Heavy ion data is not discussed here but can be found
in references \cite{The_ATLAS_Collaboration_2008}\cite{Aad_2024}. The fration of
operational channels for the end of Run 2 was $99.5\%$ for calorimetry and
$95\%$ for tracking systems. Over eight years of data taking from the start of
Run 1, the ATLAS data taking efficiency rose from $90\%$ to $95\%$. The fraction
of this data collected which is marked as good for physics arose from $88.8\%$
to $97.5\%$. During Run 2, the average pileup was 33.7, and for Run 3, at an
instantaneous luminosity of $\mathscr{L} = 2\times 10^{34}
\;\text{cm}^{-2}\text{s}^{-1}$, peak pile up was between $52-57$ with a maximum
of $<60$ \cite{Aad_2024}. 

Beam optics and techniques were also improved during Run 2. From Run 1 to Run 2,
a larger number of bunches fro a shorter bunch spacing at higher energy were
used. This led to ``brighter'' collisions. This is attributed to the PS
injection scheme, which was changed in 2016, which was called Batch Production
Merging and Splitting (BPMS). The result was a smaller beam size in the
transverse plane. In 2017, the crossing angle anti-legelling scheme was
introduced, which gradually reduced the crossing angle at the injection point as
the bunch intensity decreased. In 2018, this was augmented with a $\beta*$
anti-levelling scheme, which gave an approximate $15\%$ additional squeeze at
the injection point at the end of each fill. The result of both schemes inreased
the integrated luminosity without increasing the peak instantaneous luminosity
by running for longer near the peak instantaneous luminosity \cite{Aad_2024}.

\begin{wrapfigure}{r}{0.5\textwidth}
    \centering
    \includegraphics[width=0.5\textwidth]{atlas_schematic}
    \caption{Schematic of the ATLAS detector.}
    \label{fig:atlas_schematic}
\end{wrapfigure}

\newpage
\begin{figure}[t!]
    \centering
    \begin{subfigure}[t]{.48\textwidth}
        \centering
        \includegraphics[width=\columnwidth]{intlumivstimeRun2}
        \caption{Luminosity collected from the full LHC Run 2.}
        \label{subfig:lumi_run2}
    \end{subfigure}
    \hfill
    \begin{subfigure}[t]{.48\textwidth}
        \centering
        \includegraphics[width=\columnwidth]{intlumivstimeRun3}
        \caption{Luminosity collected from the partial LHC Run 3.}
        \label{subfig:lumi_run3}
    \end{subfigure}
\caption{Luminosity delivered to ATLAS from the LHC for the duration of data taking for the hh4b analysis.}
\label{figure:lhc_luminosity}
\end{figure}

\section{Research Contributions to ATLAS}

\subsection{Machine Learning for Pions}
My first research project within ATLAS as with the Machine Learning for Pions
project. At ATLAS, an inportant step in physics analysis is the reconstruction
of incident particle energy. One decay mechanism for a particle with some high
initial energy is to form energetic particle showers within the detector. Thus
with the primary goal in mind of reconstructing fundamental interactions, we
seek to accurately reconstruct the total energy of particle showers, henceforth
referred to as \textit{jets}. The Machine Learning for Pions project had a goal
of investigating if machine learning techniques could improve not only energy
reconstruction but also incident particle classification as well. This was done
with Pions due the fac tthat as a light hadron, they are produced in abundance
at the LHC and are a primary reconstruction object within a jet. This research
project was largely a success, demonstrating that deep learning was able to
better reconstruct energies of jet substructure elements when compared with the
standard technique used within ATLAS of Local Cell Weighting (LCW). Moreoever,
we were able to demonstrate the use of complimentary information from the
detector performed better than using this information on an individual basis.
See the appendix \ref{subfig:complimentarity} for the relevant plot.

\subsection{Time at CERN}
During my PhD I was given the opportunity to travel to the LHC and work on site
at CERN. This was a valuable opportunity to collaborate in person with
researchers from my team on various projects. In addition, I was able to take
part in underground visits to see the ATLAS detector in person. Beyond seeing
the detector in person, I was able to get trained and take shifts in the ATLAS
control room. My responsibilities involved two positions, the Run Controller,
and the Trigger Shifter. The Run controller is responsible for starting and
stopping data daking runs, and choosing which subcomponents of the detector are
involved in recording data. The trigger shifter is responsible for monitoring
data rates, CPU processing usage, and controlling which events we will record,
as dictated by the \textit{trigger menu}.

\subsection{Authorship Qualification Project}
Active collaborators with the ATLAS experiment must complete some task that is
needed by the experimental apparatus (or software) in order to qualify for
authorship on papers. For this task, I worked on studies related to an important
step in data processing called \textit{jet cleaning}. When we take data with
ATLAS, not all of the jets which we record are from the physics events we are
interested in. Jets which are reconstructed which are \textit{not} from
proton-proton collisions are referred to as \textit{fake jets}. There are three
main sources of fake jets. These are calorimetry noise, beam induced background
(BIB) or events from cosmic muons causing showers in the detector. In order to
remove these events, we make cuts based on intuitive physics. The strength of
cuts are fixed at given \textit{working points}. It was my task to investigate
if these cuts were working as expected for the newest software releases, and
characterize their performance. In addition, if possible these cuts were to be
improved upon.

The performance of these cuts were suitably characterized across the geometry of
the detector, as well as with respect to the transverse jet momentum.
Efficiencies were studied for all working points from data from both Run 2 and
Run 3 with Athena Release 22.

See \ref{fig:cleaning_performance_subfigure} for the full characterization.
\newpage

\newpage
\section{Resonant Analysis of Boosted Di-Higgs to Four Bottom Quarks}
\subsection{Theoretical Motivation}
\begin{figure}[t]
    \centering
    \begin{subfigure}[t]{.48\textwidth}
        \centering
        \includegraphics[width=\textwidth]{RSG_resonant_modes.png}
        \caption{
        Visual representation of the resonant modes resulting from the solution
        of the equation in warped extra dimensions. The resonant modes are
        pushed out to the boundary which is the energy limit for the physics we
        wish to study.
        }
        \label{subfig:RSG_resonant_modes}
    \end{subfigure}
    \hfill
    \begin{subfigure}[t]{.48\textwidth}
        \centering
        \includegraphics[width=\textwidth]{warp_factor}
        \caption{Graphic representation of the relative strength of gravity
        spreading out at the boundary.}
        \label{subfig:warp_factor}
    \end{subfigure}
\caption{}
\label{fig:theoretical_motivation_figure}
\end{figure}
The Higgs boson gives us a window to probe new physics from multiple
interpretations. There are two primary mechanisms that will be discussed.
However, this thesis intends to keep the search as general as possible such that
it is as model agnostic as possible. Unless there is a search for physics that
is extremely well motivated, it is always best to keep the search as model
agnostic as possible. The Higgs Boson is an interesting particle to do a model
agnostic search because there is such a rich ecosystem of physics beyond the
standard model that can (\textit{and should}) in principle couple to the higgs.
It is in the experimentalists best interests with respect to time and energy
investment to perform a search that can capture as much of this ecosystem as
possible. With that in mind, there is such an idea that has not yet been ruled
out experimentally and is very well motivated by one of the largest questions in
theoretical physics. That is, namely the \textit{big hierarchy problem}. As we
shall see, the phenomenology of such a model couples to the Higgs. Thus, as a
primary goal the aim is to search for the resulting particle.

The phenomenology in question is that of \textit{extra dimensions}. This idea is
not new, and in fact has been around for a long time, since the 1920s. There are
different, yet equally valid geometric configurations that one can come up with
in order to add dimensions. For simplicity, we will discuss 5-D goemetries where
we have four spatial dimensions and one of time. Extra dimensional models can of
course be extendid in the same way much beyond this but for now we will stick
with the simplest case.

The obvious extension is that of flat extra dimensions. Say more about flat
extra dimensions

A slightly more complex geometry is that of a curled extra dimension. The
important features of this model are that the four ``normal'' dimensional space
form flat Minkowski space, and the extra spatial dimension forms a closed loop.
This creates a closed boundary condition, which allow for resonances.

One new twist that has been added to this idea is that the dynamics are modified
as a parameter of the extra dimension. This was proposed by Lisa Randall and X
Sundrum, and is called the Randall-Sundrum geometry. There are two distinct
configurations of this geometry each with their own phenomenological
consequences. As we shall see from investigating the 5-D action potential, the
reasons for this being an interesting theory to investigate are twofold. For
one, it can provide a natural explanation for that hierarchy problem. That is,
why is gravity so weak in comparison to other forces? Also, we will see that
this model predicts resonant modes that are not evenly spaced out, but rather
bunched together closer to the standard model energies.

The five dimensional action potential can be expressed as the following
\begin{equation}
    S_{\text{Higgs}} = \int
    \dd[4]{x}\sqrt{-g^{(2)}} \left[g^{(2)\mu\nu}\pd_\mu\pd^\nu \right]
\end{equation}

\subsection{Boosted Topology}
Heavy particles will decay to highly energetic states due to mass energy
equivalence. For a di-Higgs state, this means each higgs will have high
momentum. The highest branching fraction for the higgs is approximately 57\% to
bottom quarks. Therefore, with the final state of two bottom quarks, the two b
quarks will subsequently have high momentum also. Thus, for a heavy resonant
search, we seek a topology with two bottom quarks kinematically favoring the
direction of the boosted higgs. At a certain energy, the b quark jets are highly
overlaping, and they are better reconstructed as a single large radius jet with
``two pronged'' topological characteristics. Jets selected for this purpose are
$R=1.0$ Unified Flow Object jets \ref{ufo_jets}.
\begin{figure}[t]
    \centering
    \begin{subfigure}[t]{.48\textwidth}
        \centering
        \includegraphics[width=\columnwidth]{Boosted_Topology}
        \caption{Boosted signal topology showing the two bottom quarks forming as a single large radius jet.}
        \label{subfig:boosted_topology}
    \end{subfigure}
    \hfill
    \begin{subfigure}[t]{.48\textwidth}
        \centering
        \includegraphics[width=\columnwidth]{Boosted_Topology}%{Feynman Diagram}
        \caption{Feynman diagram showing the scattering process.}
        \label{subfig:feynman}
    \end{subfigure}
\caption{Showing the signal topologies}
\label{fig:signal_topologies}
\end{figure}
\newpage

\subsection{Xbb Tagging}
\begin{figure}[t]
    \centering
    \includegraphics[width=0.75\textwidth]{Figure2_GN2X_Tagger_PubNote}
    \caption{Performance of the Xbb tagger.}
    \label{fig:Xbb_tagger}
\end{figure}

For many physics objectives within ATLAS, it is important to identify
$H\rightarrow b\bar{b} / c\bar{c}$ processes. ATLAS has implemented the GN2X tagger
for identifying jets originating from such events. In the boosted regime, we use
the GN2X tagger for identifying large radius jets. The GN2X has been able to
identify a significant improvement over the previous approach. The previous
approach used flavour tagging discriminants of individual track-based subjets in
a basic feed forward neural net.  Contrast this to the GN2X which builds upon
previous results of Graph Neural Net (GNN) networks to tag and calibrate small
radius jets, and adds transformer architecture. In addition there are some
variations that have been explored combining tracks, subjets, and large R jet
calorimetry constituents to increase the performance.

GN2X is trained with the goal of classifying large radius jets based on their
origin, that is to discriminate $H\rightarrow b\bar{b}/c\bar{c}$ from background
processes that could produce jets of a similar structure (and substructure).
Backgrounds considered in training were just two - multijet and fully hadronic
top quark decays. Examples of some substructure elements that the tagger would
``learn'' are topological characteristics such as displaced secondary tracks for
b-jets. In addition it may observe the two-pronged nature of the large radius
jet, as it is composed of two smaller sub-jets. Subjects within the large radius
jet are defined with a variable radius clustering algorithm
\cite{jet_substructure}.

The three main samples used in training are for the signal and two backgrounds.
For the H jet signal, \textit{associated Z production} higgs samples were
produced. That is, we have $q\bar{q}\rightarrow ZH, Z\rightarrow \mu^+\mu^-$.
For top quark jet backgrounds, the process $Z'\rightarrow t\bar{t}$ is used.
Multijet is simply produced in the standard QCD multijet production within
ATLAS. One difficulty of this training, is that with a standard Higgs
production, we will have a distribution centered at $125 \text{GeV}$, and any
neural network will simply adjust the weights and biases to select the 125 GeV
sample. To overcome this, the phase space sampling is biased such that the
tagger is trained on a flat mass distribution. This becomes of utmost importance
when GN2X is performing simultaneous mass regression for the jet mass
calibration. However, when it comes to evaluation of the network, a separate
sample of associated higgs production is used with H set to $125 \text{GeV}$ and
a smooth, exponentially falling distribution for $p_T$.

For the signal sample, associated ZH Higgs production is used. For top quark
backgrounds, a $Z'$ production is used. Finally, for multijet the standard ATLAS
QCD multijet production is used. Productions are summarized in the following
table.
\begin{table}[h]
    \centering
    \begin{tabular}{ c c }
        \noalign{\vskip0pt}\hline\noalign{\vskip3pt}
        $H(b\bar{b}/c\bar{c})$ & $q\bar{q} \rightarrow ZH,\; Z \rightarrow
        \mu^+\mu^-$ \\ 
        Top & $Z' \rightarrow t\bar{t}$ \\
        Multijet & Multijet \\
        \hline
    \end{tabular}
\end{table}

Samples processed using the full production chain for detector simulation, using
\cite{geant4} and ATLAS Athena software \cite{atlas_simulation}. Pileup is
simulated by overlaying hard scattering events with inelastic collisions
\cite{boosted_hbbcc_tagger}. Signal production composed of 62 million jets in
total. $15$ million each for $H(b\bar{b}/c\bar{c})$, $10$ million Top jets, and
$22$ million multijets.

UFO Jets merge Particle Flow \cite{pflow_jets} objects with Track Calo Clusters
\cite{calo_clusters_performance}. This is a combination of neutral and charged
components, in order to optimize performance for different physics objects
across a wide range of $p_T$. Pileup, and other contributions were removed with
grooming algorithms; soft-drop \cite{soft_drop}, constituent subtraction
\cite{pileup_subtraction}[?], and soft-killer \cite{softkiller}. The jet
selections applied are summarized in the following table.

\begin{table}[h]
    \centering
    \begin{tabular}{ c c c c }
        \noalign{\vskip0pt}\hline\noalign{\vskip3pt}
        & Energy & Jet Mass & $|\eta|$ \\
        Training & $200\;\text{GeV} \rightarrow 1.5\;\text{TeV}$ &
        $50\;\text{GeV} < m_J < 300\;\text{GeV}$ (flat) & $< 2.0$ \\ 
        Evaluation & $250\;\text{GeV} \rightarrow 1.5\;\text{TeV}$ &
        $50\;\text{GeV} < m_J < 200\;\text{GeV}$ & $< 2.0$ \\ 
        \hline
    \end{tabular} 
    \caption{Traning and evaluation summary for input jets \cite{boosted_hbbcc_tagger}.}
    \label{tab:jets_training}
\end{table}

Truth information is recorded from the generator level. It is a requirement that
the large radius jets have a ghost associated Higgs. Similarly for the top quark
background, any $t\bar{t}$ particle that decay hadronically that have a ghost
association are kept.

Training inputs are mostly composed of tracking variables. However, three
large-R jet variables are considered - $p_T, \eta, m_J$. $20$ track variables
associated with each track are considered, and for each jet, up to $100$ tracks
are considered.

GN2X uses a similar architecture to deep set neural networks CITE for pre-processing
of the track vectors. That is, track variables have initializer networks which
embed the tracking information as a higher dimensional vector. Variants were
explored where different inputs have different initializer networks. This higher
dimensional embedding is a $192$ dimensional space. Crucially, what sets apart
the performance of GN2X from previous ML models, is the learnable aggregation.
This is architecture that uses information from the embedded dimension, to
optimize important features for global message passing, and aggregation. This is
done with transformer-encoder architecture. GN2X was build with 6 encoder and 4
attention layers. Heterogeneous models which do not differentiate between input
variants use the same attention mechanism for all inputs, and this significantly
reduces the overall model parameter size \cite{boosted_hbbcc_tagger}.

The output of these networks is probability distributions for given tagging
category. The efficiency of the tagger is defined as the number of correctly
identified $H(bb)$ jets over all $H(bb)$ jets. The descriminant for the tagger
is defined as
\begin{equation}
    D^{GN2X}_{Hbb} = \ln \left( \frac{p_{Hbb}}{ f_{Hcc}p_{Hcc} +
    f_{top}p_{top} + ( 1 - f_{Hcc} - f_{top} )p_{QCD} } \right)
\end{equation}
$f_{top},\;f_{Hcc}$ are parameters allowed to float in order to control the
relative weights of each process. For the final results, weights of these
parameters are set to $f_{Hcc} = 0.02, \; f_{top} = 0.25$ arbitrarily.

The final results are shown in \ref{fig:Xbb_tagger}. Accross the whole spectrum
of tagging efficiencies the GN2X outperforms the previous tagging algorithm with
respect to the number of backgrounds rejected. This is computed for the two main
sources of background - Top quark and QCD multijet. If the results are
normalized to the previous benchmark - the simple feed-forward DNN, we are able
to see the increased ratio of the background rejection for the new algorithm.
From the QCD multijet backgorund, we see an approximately constant rejection of
$2.5$ accross the spectrum of efficiencies. For Top quark rejection, we see a
maximum at approximately $2.1$ at $.7$ tagging efficiency, with a slow decrease on either
side. We know that QCD is the dominant background, and this background rejection
matters more. Given the increase in overall tagging efficiency, adding the GN2X
to our analysis should yield more sensitive results than previos analysis.

\vspace{10pt}
DEPRECATED
For tagging signal and discriminating background we use the state of the art
GN2X tagger. This is a multi-headed attention network. GN2X\* is trained on
truth boosted Xbb jets. Therefore in principle the truth information that the
network is exposed to has the two-pronged topology. This tagger also learns the
truth jet reconstructed mass and energy. A flaw 

Finally, the results are such that accross the whole spectrum of working points
for $H(b\bar{b})$ tagging efficiency, the GN2X algorithm outputforms the
previous iterations of taggers. Namely the $D_Xbb$, which is simply the feed
forward neural net. The final results are summarized in Figure
\ref{fig:Xbb_tagger}.

\newpage
\subsection{Previous Results}
\begin{figure}[t]
    \centering
    \begin{subfigure}[t]{.48\textwidth}
        \centering
        \includegraphics[width=\textwidth]{previous_results_scalar}
        \caption{Previous results with 2018 data for a generic scalar.}
        \label{subfig:previous_results_scalar}
    \end{subfigure}
    \hfill
    \begin{subfigure}[t]{.48\textwidth}
        \centering
        \includegraphics[width=\textwidth]{previous_results_graviton}
        \caption{Previous results with 2018 data for a spin-2 particle.}
        \label{subfig:previous_results_graviton}
    \end{subfigure}
\caption{Past results with resonant analysis for full Run 2 data.}
\label{fig:previous_results}
\end{figure}

Previous heavy resonance searches have been conducted with ATLAS in the Di-Higgs
to four b channel for both boosted and resolved channels. These analysis use
$126 \text{fb}^{-1}$ of data for resolved and $139\text{fb}^{-1}$ for boosted.
This is data from Run 2 operating at $\sqrt{s}=13\text{TeV}$. Both channels
trigger on jets reconstructed with the anti-$k_t$ algorithm \cite{antikt}.
Spin-0 and Spin-2 samples were produced at leading order with MadGraph5
\cite{madgraph5_OG} and simulated with Geant4 \cite{geant4}. Physics objects are
reconstructed with Athena \cite{atlas_simulation}.

This analysis used a different signal, control and validation region than the
the Vector Boson Fusion (VBF) paper \cite{vbf_hh4b}, and current work. The
selections are as follows
\begin{align}
    SR &= \sqrt{\left(  \right)^2 + \left(  \right)^2} \\
    VR &= \sqrt{\left(  \right)^2 + \left(  \right)^2} \\
    CR &= \sqrt{\left(  \right)^2 + \left(  \right)^2} \\
\end{align}

No significant excesses were observed, and there was good agreement between the
background estimation and observed data. Profile likelihood fits were carried
out in bins of $M_{HH}$ for both boosted and resolved channels. Global
significance is computed using the number of crossings below $p=0.5$ is used in
combination with the local $p-\text{value}$ to produce a global
$p-\text{value}$. The most significant excess is found for a signal with mass of
1100 GeV, which had a local significance of $2.3\sigma$. The global significance
was found to be $0.4\sigma$ for the spin-0 resonance and $0.8\sigma$ for spin-2.

Upper limits on produciton cross section for gluon-gluon fusion ($\sigma_{ggF}$)
were produced based on the $CL_S$ method, set at the 95\% confidence level. The
results can be seen in Figure \ref{fig:previous_results}. Below $3 \text{TeV}$,
limits are computed with asymptotic formulae. Beyond this mass point, this
approimation is innacurate and thus limits are computed with sampleing
psuedo-experiments. In Figure \ref{fig:previous_results}, a model with a base
case of $k/\overline{M}_{Pl}=1$ is shown. This is an arbitrary number, however
the motivation is such that extra dimensions can explain the \textit{big
hierarchy} problem with little fine tuning, and thus one would want
\textit{natural} values. This is of course a matter of taste. For this choice,
masses between the range of $298\;\text{GeV}$ and $1460\;\text{GeV}$ are
excluded \cite{hh4b_resonant}.

\newpage
\subsection{My Contributions}

\begin{figure}[t]
    \centering
    \begin{subfigure}[t]{.48\textwidth}
        \centering
        \includegraphics[width=\columnwidth]{jet_reconstruction}
        \caption{Mass reconstructrion of large radius jets from the Randall-Sundrum graviton sample.}
        \label{subfig:jet_reconstruction}
    \end{subfigure}
    \hfill
    \begin{subfigure}[t]{.48\textwidth}
        \centering
        \includegraphics[width=\columnwidth]{boosted_m_hh-2023-ROI_incl_SR-2tag_wp85}
        \caption{Input histograms for the statistical fits.}
        \label{subfig:input_hists}
    \end{subfigure}
\caption{My contributions}
\label{fig:my_contributions}
\end{figure}
Hypothesis testing requires a signal. Therefore, first steps in my contribution
to this analysis were with respect to signal production. Signal production is
generally handled by the Paricle Modelling Group (PMG) within the
collaboration. Old signal production files existed in the archived PMG central
GitLab repository and thus I was able to use a baseline production file. There
were some simple modifications required in order to scale the signal mass width.
Spin-2 graviton signals were produced with MadGraph5, and particle showers
simulated with Pythia 8. The truth Monte-Carlo information is then run through
the full detector simulation with Athena. This procedure was extended to higher
mass ranges than previous samples - 26 all together, from 300 to 6000 GeV. Steps
in signal mass production are gradually increased over the range, which
approximately respects the exponential decrease in background statistics
throughout this range. The full suite of kinematic variables for both particle
flow, and UFO jets were plotted for all mass points. These kinematics histograms
have been inspected for deficiencies and approved by the analysis team.

Monte-Carlo simulation of QCD Multijet backgrounds relies on a specific process
in order to get a physically representative result. It is difficult to produce a
specific number of jets at a particular $p_T$, and therefore it is more
computationally efficient to produce the minimum number events needed and
\textit{reweight} them, in order to get a smoothly falling $p_T$ distribution.
This is done in ten separate ``slices'' of $p_T$. Early studies involved
investigating the kinematic distributions of each $p_T$ slice individually as
well as combined to ensure that we are accurately reconstructing the Monte-Carlo
data.

In addition, other early contributions to the analysis involved comparing the
Monte-Carlo simulated backgrounds with blinded data. This involed studying all
possible permutations of GN2X working points as well as selections, and
investigating if there was closure for the kinematic properties. Moreover, we
wanted to study the overall yields of Monte-Carlo in comparison with the data. Although our
studies demonstrated that \textit{some years of data} had good agreement with
the tighest selections, the general conclusion was that the Monte-Carlo
backgrounds were insufficient for the analysis. The trend appeared to show that
recent data taking years are better with respect to both shape and yields.
However, our results are unsurprising, as it is generally known within ATLAS
that multijet backgrounds are not well modelled due to the intrinsic
complexity of quantum chromodynamic processes.

As we are primarily studying two jets for the boosted analysis, a central tool
for making region of interest selections comes from studying \textit{the mass
plane} formed with the two jets. As our ROI selections are simply a function of
jet mass, producing mass plane plots is important to visualizing, understanding,
and optimizing our signal region definitions. Thus far I've written analysis
code to produce these two dimensional histograms and superimpose the signal
region selections. These roi selections are still being studied as to what will
provide the optimal sensitivity.
The current strategy for ROI is based on the Vector Boson Fusion (VBF) paper
\cite{vbf_hh4b}.
\begin{align}
    SR &= \sqrt{\left(  \right)^2 + \left(  \right)^2} \\
    VR &= \sqrt{\left(  \right)^2 + \left(  \right)^2} \\
    CR &= \sqrt{\left(  \right)^2 + \left(  \right)^2} \\
\end{align}

Crucially, I've written code to pass through the full selection pipeline for all
the various sources of data that we have studied, namely blinded data, signal,
monte-carlo background, and the current two candidates for background estimates.
As each data type is loaded, cut selection efficiences are produced and saved.
Finally, a histogram production pipeline has been produced in order to
selectively load given data sources, make the appropriate cuts, and produce
histograms for the signal, data and background in order to pass to the
statistial fitting framework. This has been done with a command line interface

Finally, as part of the statistical analysis, I have built a pipeline which
takes the histograms and applies the CLS fits to produce exclusion limits.
Unfortunately, the chosen statistical framework (Cabinetry) \cite{cabinetry}
does not handle configurations for multiple fits for many signals. Part of my
recent work has been building and maintaining a wrapper around the statistical
tools such that we can adjust the configuration files on the fly, and run
separate fitting processes for each mass point.

At this point, pushing the resonant interpretation forward is primarily my
responsibility for this analysis.

% \begin{itemize}
%     \item [x] Signal production, JIRA request, post processing
%     \item [x] Data Monte Carlo comparisons
%     \item [x] QCD Background vefification
%     \item [x] Mass plane plots
%     \item [x] Data selection pipeline and summaries
%     \item [x] Histogram production with CLI tool
%     \item [x] Bilding and maintining collaborative tools for stats framework
%     \item Resonant interpretation is nearly entirely my responsibility
% \end{itemize}

\newpage
\clearpage

\subsection{Preliminary Results and Ongoing Studies}
\begin{figure}[t]
    \centering
    \begin{subfigure}[t]{.48\textwidth}
        \centering
        \includegraphics[width=\columnwidth]{Run2-CLS_fits-m_hh}
        \caption{Fits from Run 2 data.}
        \label{subfig:cls-run2}
    \end{subfigure}
    \begin{subfigure}[t]{.48\textwidth}
        \centering
        \includegraphics[width=\columnwidth]{Run3-CLS_fits-m_hh}
        \caption{Fits from Partial Run 3 data.}
        \label{subfig:cls-run3}
    \end{subfigure}
\caption{Comparison of CLS fits for Run 2 and partial Run 3 data.}
\label{fig:cls_fit_results}
\end{figure}
Thus far we have put together the majority of the pipeline for the analysis.
Preliminary studies have been done of Data/MC comparisons. Signals have been
produced and are ready for hypothesis testing. Early background estimates are
complete. We have studied our Signal, Validation, and Control regions through
the 2-D mass plane and parameterized selections of the ROI. Cuts, data
selections and working points are being investigated as to what may provide the
optimal sensitivity for our statistical fits. Histogram overlays of signal and
backgrounds are produced as inputs for our statistical fitting framework.
Finally, the fitting framework is producing meaningful limits for each mass
point. Ongoing work is being done in order to scale the signal inputs
appropriately for each mass point such that we have a fixed production cross
section in order to have a reference point for our cross section exclusion
plots, produced by the CLS fitting limits. This is shown in Figure
\ref{fig:cls_fit_results}. 

\section{Conclusion}
For this analysis, we have a very well motivated theoretical model to study. In
addition, there are many BSM models which couple to the Higgs. Therefore, it is
strongly motivated to do generic heavy resonance searches, in addition to the
Graviton search. This study can be added with little additional effort, beyond
the signal request and preliminary data processing, which has negligeable
processing time. As we are building upon the previous searches done by others
\cite{previous_results}, we have a good understanding of many aspects of the
analysis. These include, signal topology, background estimation techniques,
signal, validation and control region selections, as well as a reference for
sensitivity. There are several improvements to the general analysis techniques
that give reason to believe that these results will be more sensitive than
previous results. First we have a considerably more data. In addition to
additional integrated luminosity, the $pp$ collisions are occuring at a higher
center of mass energy. Although this is an incremental increase, it will have a
disproportionate effect on heavy resonance searches. Furthermore, the
improvements to Xbb tagging (a roughly $2.5\times$ improvement), will yield a
higher signal sensitivity than before. In addition, there have been improved
techniques to large radius jet reconstruction and calibration since the previous
results were published \cite{large_r_jet}. The preliminary results are promising
thus far, and the analysis is proceeding within a reasonable timeline for
graduation.


\newpage
\appendix{Glossary}
\section{Glossary}
\subsection{Acronyms}
\acronym{AI}{Artificial Intelligence}

\acronym{GPU}{Graphics Processing Unit}

\acronym{ML}{Machine Learning}

\subsection{Terms}
\term{Algorithm}{A step-by-step procedure for solving a problem}

\term{Database}{A structured collection of data}

\term{Variable}{A symbol that represents a value that can change}

\appendix{Machine Learning for Pions}
\begin{figure}[t]
    \centering
    \begin{subfigure}[t]{.48\textwidth}
        \centering
        \includegraphics[width=\textwidth]{EnergyIQROverMedian}
        \caption{Neural network response demonstrating energy reconstruction
        median and the interquartile range.}
        \label{sufig:ml4p_energy_response}
    \end{subfigure}
    \hfill
    \begin{subfigure}[t]{.48\textwidth}
        \centering
        \includegraphics[width=\textwidth]{clusters_tracks_combined}
        \caption{Figure demonstrating the energy response complementa}
        \label{subfig:complimentarity}
    \end{subfigure}
\caption{Main results from the machine learning for pions project.}
\label{fig:ml4p_main_results.}
\end{figure}
\appendix{The ATLAS Detector}
\begin{figure}[t]
    \centering
    \begin{subfigure}[t]{.48\textwidth}
        \centering
        \includegraphics[width=\textwidth]{Inner_Detector_Schematic}
        \caption{Schematic of the ATLAS Inner Detector.}
        \label{subfig:inner_detector_schematic}
    \end{subfigure}
    \hfill
    \begin{subfigure}[t]{.48\textwidth}
        \centering
        \includegraphics[width=\textwidth]{ATLAS_DAQ}
        \caption{ATLAS DAQ overview.}
        \label{subfig:daq_schematic}
    \end{subfigure}
    ~
    \centering
    \begin{subfigure}[t]{.48\textwidth}
        \centering
        \includegraphics[width=\textwidth]{Calo_System}
        \caption{Calorimetry System.}
        \label{subfig:calo_schematic}
    \end{subfigure}
    \hfill
    \begin{subfigure}[t]{.48\textwidth}
        \centering
        \includegraphics[width=\textwidth]{Muon_System}
        \caption{Calorimetry System.}
        \label{subfig:muon_system_schematic}
    \end{subfigure}
\caption{Schematics for the ATLAS detector main subsystems \cite{Aad_2024}.}
\label{fig:ATLAS_Schematics}
\end{figure}

\newpage
\begin{figure}[t]
    \centering
    \includegraphics[width=\textwidth]{ATLAS_DAQ}
    \caption{ATLAS DAQ overview. \cite{Aad_2024}}
    \label{fig:daq_schematic}
\end{figure}

\newpage

\appendix{Qualification Project}
\begin{figure}[t]
    \centering
    \begin{subfigure}[t]{.45\textwidth}
        \centering
        \includegraphics[width=\textwidth]{GoodJets_FracSamplingMax_EMFrac}
        \caption{Cleaning performance mapped with respect to pt and some other stuff.}
        \label{subfig:tight_th2}
    \end{subfigure}
    \hfill
    \begin{subfigure}[t]{.45\textwidth}
        \centering
        \includegraphics[width=\textwidth]{FakeJets_FracSamplingMax_EMFrac}
        \caption{Cleaning performance mapped with respect to pt and some other stuff.}
        \label{subfig:tenacious_th2}
    \end{subfigure} %
\caption{Jet Cleaning figures}
\label{fig:cleaning_performance_subfigure}
\end{figure}

\appendix{Boosted Resonant Di-Higgs to Four b}
\begin{figure}[t]
    \centering
    \begin{subfigure}[t]{.48\textwidth}
        \centering
        \includegraphics[width=\textwidth]{graviton_overlay}
        \caption{figure caption}
        \label{subfig:graviton_overlay}
    \end{subfigure}
    \hfill
    \begin{subfigure}[t]{.48\textwidth}
        \centering
        \includegraphics[width=\textwidth]{jet_reconstruction}
        \caption{Reconstructed jet mass for large radius (R=1.0) UFO jets from
        the Randall-Sundrum Graviton signal.}
        \label{subfig:ufo_rsg_reconstruction}
    \end{subfigure} %
\caption{Some generic caption.}
\label{fig:some_generic_figure}
\end{figure}

\newpage
\newpage
\appendix{Derivation of RS Model}
\section{Dynamics in Four Dimensions}
First let use make some fundamental observations about allowing for extra
dimensions. Momentum in a five (or higher) dimensional spacetime can show up as
extra mass in four-space. 
\begin{align}
    E = \sqrt{\vb{p}^2 + p_5^5 + m^2} \\
    E = \sqrt{\vb{p}^2 + m_{KK}}
\end{align}
We make the definition that $x^5 = (x^\mu,x^5)$ where $(M = 0-3,5)$. For
compactfied extra dimensions (curled, closed) we confine $x^5 \equiv \gamma$ to
the interval $0 \leq \gamma < 2\pi R$. Also we consider that the geometry is
periodic in nature and therefore scalar fields will obey the following
\begin{equation}
    \phi(x, \gamma + 2\pi R) = \phi (x,\gamma)
\end{equation}
With a simple five dimensional metric, we consider the space-time element is the
following
\begin{equation}
    \dd{s}^2 = \eta_{MN}\dd{X}^M\dd{X}^N = \eta_{\nu\mu}\dd{x}^\mu\dd{x}^\nu -
    \dd{\gamma}^2
\end{equation}
where we have used the metric signature $\nu_{MN} = (1,-1,-1,-1,-1)$.
We can define the action for a 5-D scalar with no mass to be the following
\begin{align}
    S^{5-D} &= \int \dd[5]{X} \left( \frac{1}{2} \pd_M \Phi(X) \pd^M
    \Phi(X) \right) \\
        S^{5-D} &= \int \dd[4]{x} \int^{2\pi R}_{0} \left[ \frac{1}{2} \pd_\mu \Phi(x,\gamma) \pd^\mu
    \Phi(x,\gamma) + \pd_\gamma \Phi(x,\gamma) \pd^\gamma
    \Phi(x,\gamma) \right]
\end{align}
Based on the boundary conditions, we know that the field will be periodic and we
can therefore expand any wavefunciton as a fourier series
\begin{align}
    \Phi(x,\gamma) &= \frac{1}{s\pi R} \sum^{+\infty}_{n=-\infty}
    \varphi^{n}(x)e^{in\gamma /R} \\
                   &= \frac{1}{2\pi R} \Phi^{(0)}(x) + \frac{1}{\pi R}
                   \sum^{+\infty}_{n=-\infty} \left[
                       \phi^{+(n)}(x)\cos\left( \frac{n\gamma}{R} \right) +
               \phi^{+(n)}(x)\sin\left( \frac{n\gamma}{R} \right) \right]
\end{align}
Imposing the condition that the field $\Phi$ must be real, means that
$\phi^{(-n)} = \phi^{(n)\dagger}$. We can define the field
\begin{equation}
    \phi^{(\pm n)}(n > 0) = \frac{1}{\sqrt{2}} \left( \phi^{(+n)} \pm i\phi^{(-n)} \right)
\end{equation}
If we combine these equations, we are left with an \textit{effective} 4-D action
\begin{align}
    S^{(4D)} &= \int \dd[4]{x} \frac{1}{2} \left\{ \sum_n \left[
    \pd_\mu\phi^{(-n)}\pd^\mu\phi^{(n)} - \frac{n^2}{R^2}\phi^{(-n)}\phi^{(n)}
\right] \right\} \\
             &= \int \dd[4]{x} \frac{1}{2} \left\{
                 \pd_\mu\phi^{(0)}\pd^\mu\phi^{(0)} \sum_{n=1}^{\infty} \left[
    \pd_\mu\phi^{(\pm n)}\pd^\mu\phi^{(\pm n)} - \frac{n^2}{R^2}\phi^{(\pm
    n)}\phi^{(\pm n)}
\right] \right\} \\
\end{align}
In 4-D, we have acquired a term
\begin{equation}
    \frac{n^2}{R^2}\phi^{(\pm n)}\phi^{(\pm n)}
\end{equation}
This behaves exactly like an extra mass term in four-space. Interestingly, if we
add extra compactified dimensions, we get more mass terms. For every n, we see a
mass resonance. Note that the scale of the mass depends on the geometry of the
compactified extra dimensions. Suppose we have a scalar with mass $m$. Then we
have $$m^2 = m_0^2 + \frac{n_5^2}{R_5^2}$$. Notice that for large $\frac{n}{R}$
we will have no access to observe these terms. The possible mass terms that are
accessible with accelerator energies depend on $\frac{n}{R}$.

\section{RS1}
We start with the effective four dimensional higgs action
\begin{equation}
    S_{Higgs} = \int \dd[4]{x}\sqrt{-g^{(2)}} \left[ g^{(2)\mu \nu} \pd_\mu H
    \pd_\nu H - \lambda \left( |H|^2 - v_0^2 \right)^2 \right]
\end{equation}
where we have used the following relationship with the metric
\begin{equation}
    g^{(2)}_{\mu\nu} = e^{-2k\pi R}g^(0)_{\mu\nu}, g^{(2)} = \text{det}
    g_{\mu\nu}^(2) = (e^{-2\pi k R})^4g^{(0)} = e^{-8\pi kR}g^{(0)}
\end{equation}
$g^(2)_\mu\nu$ is the gravitational interaction term with (in this case) the
higgs, but could also be extended to any SM particle. For this to work, assume
that the $4-D$ higgs and the standard model are at the $\gamma = \pi R$ brane.
When $\gamma$, this approaches the Planck or UV brane. At $\pi R$, we have the
IR brane. Expanding the above term, we are left with
\begin{equation}
    S_{Higgs} = \int \dd[4]{x} \sqrt{-g^(0)} \left[ e^{-2k\pi R} g^{(0)} \pd_\mu
    H \pd_\nu H - e^{-4k\pi R} \lambda (|H|^2 - v_0^2 )^2 \right]
\end{equation}
Again we have used the above relationship with $g^{(2)}$. Here we can
renormalize the higgs such that $H = \hat{H}e^{k\pi R}$. We are now left with
\begin{equation}
    S_{Higgs} = \int \dd[4]{x} \sqrt{-g^(0)} \left[ g^{(0)} \pd_\mu
    \hat{H} \pd_\nu \hat{H} - e^{-4k\pi R} \lambda (|\hat{H}|^2 - v_0^2 )^2 \right]
\end{equation}
Through this renormalization, something remarkable has happened. The bare vacuum
expectation value of the higgs gets renormalized to the physical vev by the warp
factor, that is
\begin{equation}
    v \equiv e^{-k\pi R}v_0
\end{equation}

\section{RS2}
In the RS2 scenario, we can show that the ``compactification'' of the extra
dimension need not be constrained to a finite radius and thus finite volume.
This scenario has a different setup, that the IR brane is located at $\pi R$,
and $\gamma = 0$ for the Planck Brane. The following steps will demonstrate that
the KK towers are pushed off to infinity.
Consider the 0 mode, which does not depend on the usual parametrization of the
space-time coordinate $\gamma$. This is the same as stating that the zero mode
does not depend on the radius $R$ explicitly.
\begin{align}
    S_5 &= \frac{\bar{M}_5^3}{2} \int \dd[4]{x} \int^{\pi R}_{-\pi R}
    \dd{\gamma} \sqrt{-G}R^{(5)} \\
    &= \frac{\bar{M}_5^3}{2} \int \dd[4]{x} \int^{\pi R}
    \dd{\gamma} \sqrt{-g}[AR^{(4)} + ...] = S_{4 eff} + ...\\
\end{align}
Where we express the Ricci curvature in 5-D and 4-D as $R^{(5)}, R^{(4)}$
respectively. The 5D Ricci curvature can be expressed as
\begin{align}
    R^{(5)} &= G^{MN}R_{MN} = G^{MN}R^K_{MKN} = G[ \pd_M \Gamma^K_{KN} + ... ] \\
            &= G^{MN}[ \pd_M \{ G^{KL}( \pd_L G_{MN} + ... ) + ... \} ]
\end{align}
Where we have expressed the Riumann curvature tensors and Christoffel
connection in 5-D as $R^J_{MKN}$ and $\Gamma^J_{KN}$. Only $M,N = \mu\nu$
contribute to $R^{(4)}$ and terms with $M,N=5$ contribute after integration.

The result of this integral is
\begin{align}
    16\pi G_N &= \frac{2}{\bar{M}_{Pl}^2} = \frac{1}{\bar{M}^3_5} \frac{2k}{1 -
    e^{-2k\pi R}} \\
    \implies \bar{M}^2_{Pl} &= \frac{\bar{M}^3_5}{k} (1 - e^{-2k\pi R})
\end{align}
What we see from this result is that even if we extend the radius out to
infinity, we have a well defined Planck mass. Thus we have a coherent physical
theory with extra dimensions of infinite volume. 

\section{Gravitons in Extra Dimensions}
The gravitational wave equation $h_{MN}$ is the same for massless 5-D scalar
except for the normalization factor.
\begin{align}
    S_{5-Scalar} &= \frac{1}{2} \int \dd[4]{x} \int \dd \gamma \sqrt{-G} G^{MN}
    \pd_M \Phi \pd_N \Phi \\
                 &= \frac{1}{2} \int \dd[4]{x} \int \dd \gamma
                 e^{-4\sigma(\gamma)} \sqrt{-g} \left[ e^{2\sigma(x)} \pd_\mu
                 \Phi \pd^\nu \Phi - ( \pd_\gamma \Phi \pd^\gamma \Phi ) \right]
    \pd_M \Phi \pd_N \Phi \\
\end{align}
For now, we only look at the zero mode. This means we only focus on the first
term in the integral. In addition, we require the boundary condition that
$\pd\Phi(x,\gamma)=0$ at $\gamma=0, \pi R$. This is the same as imposing Neumann
boundary conditions for an open ended string. If we expand $\Phi$ in the KK
tower, that is
\begin{equation}
    \Phi(x,\gamma) = \sum_{n\neq 0} \psi_n(x) \varphi_n (\gamma) \\
    (\pd_\mu \pd^\mu + m_n^2) \psi_n(x) = 0
\end{equation}
Where the second requirement is simply a statement that each mass tower
satisfies the equation for a free massive scalar. Now, for our 5-Scalar action
\begin{equation}
    S_{5-Scalar} = \frac{1}{2} \int \dd[4]{x} \int^{\pi R}_{\pi R} \dd \gamma
    e^{-2\sigma} \Phi \left[ -m^2 \pd_\mu \pd^\mu \right] \Phi 
\end{equation}
It can be shown, see reference \ref{522}, that this leads to the following
potential
\begin{equation}
    V(z) = \frac{15k^2}{8(k|z|+1)^2} - \frac{3}{2}k\left[ \delta(z) - \delta(z -
    z_i) \right]
\end{equation}
The resonant modes solving this potential are shown in Figure
\ref{subfig:RSG_resonant_modes}.

%N% BIBLIOGRAPHY
\newpage
\bibliographystyle{plain} % We choose the "plain" reference style
\bibliography{bibliography}

\end{document}

